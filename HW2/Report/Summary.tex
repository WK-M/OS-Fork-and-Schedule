\documentclass[a4paper,11pt]{report}
\usepackage[T1]{fontenc}
\usepackage[utf8]{inputenc}
\usepackage{lmodern}
\usepackage{hyperref}

\title{Assignment 2}
\author{Kendall Molas}

\begin{document}

\maketitle

\section*{Summary}

In this assignment, I used multiprocessing to implement this reader-writer's scenario. I created an init file that would create three workers with the same program (add\_files.c) with one file that was the same and one file that was different. The file that was the same for all three workers was the datafile.dat file. The files that differed between the workers were the newx.dat file where x is replaced by 1, 2, or 3. After the three workers were made, they all tried to open the datafile.dat and create an exclusive lock to it by using flock(). Flock() utilizes the flags LOCK\_EX, LOCK\_NB, and LOCK\_UN. The program for the worker would then read the datafile.dat and newx.dat and merge the two lists together in sorted order. After the program completes this merge, it writes back to the datafile.dat and closes it. When closing the file with fclose(), the lock is released. \footnote{\url{https://linux.die.net/man/2/flock}} Each worker performs this task and this results in a datafile.dat that is the same as the ans.dat file. 

\end{document}
